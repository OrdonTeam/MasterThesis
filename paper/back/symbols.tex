\chapter*{Wykaz symboli i skrótów}
\addcontentsline{toc}{chapter}{Wykaz symboli i skrótów}

\noindent
\textbf{F} - funkcja wejściowa, \newline
\textbf{G} - wejściowa składowa dekompozycji, \newline
\textbf{H} - wyjściowa składowa dekompozycji, \newline
\textbf{k} - liczba wierszy funkcji, \newline
\textbf{MR} - macierz rozróżnialności, \newline
\textbf{m} - liczba wierszy funkcji, \newline
\textbf{n} - liczba argumentów funkcji, \newline
\textbf{R} - redukt, \newline
\textbf{w\textsubscript{i}} - i-ty wiersz funkcji,\newline
\textbf{X} - zbiór argumentów, \newline
\textbf{x\textsubscript{i}} - i-ty argument funkcji, \newline
\textbf{x\textsubscript{i\textsubscript{w\textsubscript{j}}}} - wartość i-tego argumentu dla j-tego wiersza w funkcji, \newline
\textbf{Y} - zbiór wartości funkcji, \newline
\textbf{y\textsubscript{i}} - i-ta wrtość funkcji, \newline

\cleardoublepage