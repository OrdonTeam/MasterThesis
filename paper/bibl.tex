%*******************************************************************************
% Bibliografia - spis literatury wykorzystanej przy tworzeniu pracy
%*******************************************************************************

\begin{thebibliography}{99}
\addcontentsline{toc}{chapter}{Bibliografia}
\bibliographystyle{tran}

%[1] Borowik G., Łuba T.: Fast Algorithm of Attribute Reduction Based on the complementation of Boolean Function, ch. 2, pp. 25-41, Springer International Publishing, 2014.
\bibitem{fast-algorithm} G. Borowik, T. Łuba \emph{,,Fast Algorithm of Attribute Reduction Based on the complementation of Boolean Function''}, ch. 2, pp. 25-41, Springer International Publishing, 2014.

%[2] G. Borowik, K. Kowalski: Efektywna procedura uzupełnienia funkcji boolowskich i jej zastosowaniew eksploracji danych. Przegląd Telekomunikacyjny i Wiadomości Telekomunikacyjne. 03/2015.
\bibitem{efektywna-procedura} G. Borowik, K. Kowalski \emph{,,Efektywna procedura uzupełnienia funkcji boolowskich i jej zastosowanie w~eksploracji danych''}, Przegląd Telekomunikacyjny i Wiadomości Telekomunikacyjne, 03/2015.

\bibitem{inzynierka} K. Kowalski \emph{,,Implementacja Algorytmu Uzupełnienia Funkcji Boolowskich z Wykorzystaniem Programowania Współbieżnego''}, Praca Dyplomowa Inżynierska, 2014.

%[3] Chen, D., Wang, C., Hu, Q., 2007. A~new approach to attribute reduction of consistent and inconsistent covering decision systems with covering rough sets. Information Sciences 177, 3500–3518.
\bibitem{new-reduction} D. Chen, C. Wang, Q. Hu \emph{,,A new approach to attribute reduction of consistent and inconsistent covering decision systems with covering rough sets''}, Information Sciences 177, 3500–3518, 2007.

%[4] Łuba T., Borowik G.: Synteza logiczna, Oficyna Wydawnicza PW, Warszawa 2015.
\bibitem{synteza-logiczna} T. Łuba, G. Borowik \emph{,,Synteza logiczna''}, Oficyna Wydawnicza PW, Warszawa 2015.

%[5] Łuba T., Poźniak K., Zbierzchowski B.: Redukcja i kompresja zmiennych w~syntezie funkcji generowania indeksów. Przegląd telekomunikacyjny, nr. 10, 2016.
\bibitem{redukcja-kompresja} T. Łuba, K. Poźniak, B. Zbierzchowski \emph{,,Redukcja i kompresja zmiennych w~syntezie funkcji generowania indeksów''}, Przegląd telekomunikacyjny, nr. 10, 2016.

%[6] Łuba, T., Rybnik J.:  Algorithmic Approach to Discernibility Function with Respect to Attributes and Objects Reduction, Foundations of Computing and Decision Sciences, Vol. 18, No. 3-4, 241–258, 1993.
\bibitem{algorithmic-approach} T. Łuba, J. Rybnik \emph{,,Algorithmic Approach to Discernibility Function with Respect to Attributes and Objects Reduction''}, Foundations of Computing and Decision Sciences, Vol. 18, No. 3-4, 241–258, 1993.

%[7] Sasao T.: Index Generation Functions, Logic Synthesis for Pattern Matching, EPFL Workshop on Logic Synthesis & Verification, Dec. 2015.
\bibitem{sasao-workshop} T. Sasao \emph{,,Index Generation Functions''}, Logic Synthesis for Pattern Matching, EPFL Workshop on Logic Synthesis \& Verification, Dec. 2015.

%[8] Sasao, T.: Index generation functions: Recent developments,  International Symposium on Multiple-Valued Logic (ISMVL-2011), Tuusula, Finland, May 2011.
\bibitem{sasao-recent} T. Sasao \emph{,,Index generation functions: Recent developments''}, International Symposium on Multiple-Valued Logic (ISMVL-2011), Tuusula, Finland, May 2011.

%[9] Sasao, T.:  A~Reduction Method For the Number of Variables to Represent Index Generation Functions: s-Min Method, IEEE 45th International Symposium on Multiple-Valued Logic, 164–169, 2015.
\bibitem{sasao-s-min} T. Sasao \emph{,,A Reduction Method For the Number of Variables to Represent Index Generation Functions: s-Min Method''}, IEEE 45th International Symposium on Multiple-Valued Logic, 164–169, 2015.

%[10] Sasao, T.: Memory-Based Logic Synthesis, Springer New York Dordrecht Heidelberg London, 2011.
\bibitem{sasao-synthesis} T. Sasao \emph{,,Memory-Based Logic Synthesis''}, Springer New York Dordrecht Heidelberg London, 2011.

%[11] RSES – Rough Set Exploration System, http://logic.mimuw.edu.pl/~rses/
\bibitem{rses} \emph{,,RSES – Rough Set Exploration System''} [Online] http://logic.mimuw.edu.pl/~rses/

%[12] Steinbach B., Posthoff C., Improvements of the Construction of Exact Minimal Covers of Boolean Functions, Computer Aided Systems Theory – EUROCAST 2011, Lecture Notes in Computer Science Volume 6928, pp. 272-279, 2012.
\bibitem{steinbach-posthoff} B. Steinbach, C. Posthoff \emph{,,Improvements of the Construction of Exact Minimal Covers of Boolean Functions''}, Computer Aided Systems Theory – EUROCAST 2011, Lecture Notes in Computer Science Volume 6928, pp. 272-279, 2012.

%[13] Skowron, A., Rauszer, C., 1992. The discernibility matrices and functions in information systems, in: Słowiński, R. (Ed.), Intelligent Decision Support. Springer Netherlands. volume 11 of Theory and Decision Library, pp. 331– 362.
\bibitem{skowron-rauszer} A. Skowron, C. Rauszer \emph{,,The discernibility matrices and functions in information systems''}, Słowiński, R. (Ed.), Intelligent Decision Support. Springer Netherlands. volume 11 of Theory and Decision Library, pp. 331– 362, 1992.

%[14] Ślezak, D., 1998. Searching for dynamic reducts in inconsistent decision tables,  in: Proceedings of IPMU 98, pp. 1362–1369.
\bibitem{slezak} D. Ślezak \emph{,,Searching for dynamic reducts in inconsistent decision tables''}, Proceedings of IPMU 98, pp. 1362–1369. 1998.

%[15] Wang, C., He, Q., Chen, D., Hu, Q., 2014. A~novel method for attribute reduction of covering decision systems. Information Sciences 254, 181–196.
\bibitem{novel-method} C. Wang, Q. He, D. Chen, Q. Hu \emph{,,A novel method for attribute reduction of covering decision systems''}, Information Sciences 254, 181–196, 2014.

\bibitem{unate-artykul} T. Łuba, G. Borowik, K. Kowalski, P. Pecio, C. Jankowski, M. Mańkowski \emph{,,Rola i znaczenie syntezy logicznej w~eksploracji danych dla potrzeb telekomunikacji i medycyny''}, Przegląd Telekomunikacyjny, str. 110-116, 05/2014.

\bibitem{without-matrix} M. Korzeń, S. Jaroszewicz \emph{,,Finding Reducts Without Building the Discernibility Matrix''}, Intelligent Systems Design and Applications, 2005.

\end{thebibliography}
\clearpage




%===============================================================================
