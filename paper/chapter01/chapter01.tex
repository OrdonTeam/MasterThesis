\chapter{Wstęp}

W~dzisiejszych czasach gromadzimy ogromne ilości danych.
W~związku z~tym obserwujemy rozwój dziedziny eksploracji i~analizy danych.
Pewna część tych problemów jest ściśle związana z~koniecznością precyzyjnego wyodrębniania właściwych danych z~ogromnej masy danych niepotrzebnych.
Szczególnym przypadkiem analizy danych są dane binarne, reprezentowane wektorami binarnymi o~stosunkowo dużej liczbie bitów a małej liczbie danych,
które trzeba wyselekcjonować.
Jeżeli na dodatek zbiór danych selekcjonowanych podlega częstym zmianom,
to strukturę sprzętową takich układów trzeba często zmieniać.
W~związku z~tym rozwiązaniem tego problemu są struktury FPGA z~wbudowanymi pamięciami ROM.
Ale kolejną barierą w~realizacji tych układów są ograniczenia w~nadmiernej pojemności pamięci ROM o~dużej liczbie wejść – np. 40,
co byłoby wymagane przy selekcji niewielkiej liczby wektorów (wzorców), np. 10.
Można jednak przypuszczać, że binarna tablica takich wektorów: 40 kolumn oznaczanych $x_1,…,x_{40}$ i~10 wierszy zawiera kolumny $x_i, x_j, ..., x_k$,
które ustawione obok siebie reprezentują różne liczby binarne \cite{sasao-workshop}.
Zwykle takich kolumn jest wyraźnie mniej od liczby zmiennych (bitów) wektorów selekcjonowanych.
Zatem jedną z~metod rozwiązania zadania selekcji jest redukcja argumentów,
a cały problem jest określany mianem syntezy funkcji generowania indeksów.

Zagadnienie redukcji argumentów jest tożsame z~problemem redukcji atrybutów.
W~tym ujęciu zagadnienie to było badane bardzo intensywnie \cite{fast-algorithm, efektywna-procedura, new-reduction, steinbach-posthoff, skowron-rauszer, slezak, novel-method}.
Jednym z~najbardziej znanych algorytmów redukcji atrybutów jest algorytm zastosowany w~systemie RSES \cite{rses}.
Algorytm ten został skutecznie usprawniony w~pracy inżynierskiej \cite{efektywna-procedura},
natomiast w~ramach niniejszej pracy stworzono realizację obliczającą jedno rozwiązanie [rozdz. ??].
Analogiczne zadanie podjęto w~zespole prof. Sasao,
a w~referacie \cite{sasao-workshop} znaczenie redukcji zasygnalizowano informacją,
iż opracowanie algorytmu redukcji jest jednym z~najważniejszych zadań.

Drugim aktualnym zagadnieniem syntezy logicznej w~projektowaniu generatorów adresu jest dekompozycja liniowa.
Przez dekompozycję liniową rozumie się dekompozycję $F = H(G1, G2, ...,  X)$,
w~której składowymi dekompozycji G są dwuargumentowe funkcje $EXOR$:
\begin{equation}
F = H (x_i \oplus x_j, x_k \oplus x_l, ..., X)
\end{equation}
Do najbardziej znanych algorytmów dekompozycji liniowej stosowanej w~generatorach adresów należy algorytm s-min \cite{sasao-recent, sasao-s-min}.
W~artykule \cite{redukcja-kompresja} wykazano, że algorytm s-min jest mało skuteczny i~zaproponowano całkowicie inne ujęcie tego zagadnienia.
W~niniejszej pracy metodykę tę przystosowano do heurystycznych obliczeń funkcji generowania indeksów.
Cechą charakterystyczną tej propozycji jest zastosowanie twierdzenia wiążącego dekompozycję liniową
z~tzw. zbiorem rozróżnialności [rozdz. \ref{section:single-decomposition}].
W~rezultacie opracowano metodykę oraz algorytm (i jego implementację) do heurystycznych obliczeń praktycznych funkcji indeksowania [rozdz. \ref{chapter:decomposition}].
Przykłady takich dekompozycji podano w~rozdziale \ref{section:decomposition-alorithm}.

\section{Problem generowania indeksów w~technice cyfrowej}

Problem generowania indeksów (adresu) dotyczy szczególnie tej podgrupy problemów czasu rzeczywistego,
w~których warunki podejmowania decyzji są zmienne w~czasie.
Sztandarowym zastosowaniem algorytmów generowania indeksów jest obsługa pakietów w~routerach IP.
Spotykamy się tam z~problemami filtrowania ruchu z~niepożądanych adresów (firewall) lub wybierania portów wyjściowych (routing).
W~obu przypadkach znajdują zastosowanie struktury programowalne FPGA lub EEPROM
ze względu na połączenie szybkości działania porównywalnej z~rozwiązaniami ASIC oraz elastyczności jak w~przypadku rozwiązań programowych.
Wyzwania,
jakie są stawiane przed algorytmami generowania indeksów,
są związane częstymi zmianami danych.Indeksy często muszą być generowane w~czasie rzeczywistym wraz ze zmianami danych.
Ponadto ze względu na dużą liczbę argumentów,
które muszą być brane pod uwagę (32 bity - adresy IPv4, 48 bitów - adresy mac),
wymagane są optymalizacje pozwalające wykorzystywać pamięci dostępne w~strukturach programowalnych.
Przykładowo przy funkcji o~$n=6$ argumentach wszystkich wektorów binarnych jest 64 ($2^6$).
Jeżeli spośród wszystkich wektorów poszukiwanych jest $k=6$ konkretnych, wtedy dla każdego z~64 wektorów musielibyśmy przechować 3 bity ($\log_26$).
Dla takiego przykładu minimalna potrzebna pamięć to 192 bity.
Jednak dla mechanizmu znajdowania sygnatur wirusów składającego się z~$k=1\,300\,000$ poszukiwanych wektorów o~$n=40$ argumentach, taka pamięć miałaby rozmiar 21 terabitów.
\begin{multline} \\
2^n \cdot \log_2 k = \\
= 2^{40} \cdot \log_2 1\,300\,000 = \\
= 2^{10} \cdot 2^{10} \cdot 2^{10} \cdot 2^{10} \cdot 21 = \\
=21 \cdot 2^{10} \cdot 2^{10} \cdot 2^{10} k = \\
=21 \cdot 2^{10} \cdot 2^{10} M = \\
=21 \cdot 2^{10} G = \\
=21 [Tbit] \\
\end{multline}


\section{Zastosowania}
%- Tablica trasowania
%- Terminal access controller
%- Memory patch circuit
%- Skanowanie wirusów
%- Dystrybucja adresów IP
%- Skanowanie wirusów
%- Wykrywanie niepożądanych danych
%- Konwersja kodów


\subsection{Tablica trasowania}
W~Internecie powszechne jest zagadnienie znajdowania ścieżek dla pakietów IP.
Każdy węzeł sieci przechowuje informacje dotyczące wyboru trasy dla przychodzących pakietów.
Dla każdego przechowywanego adresu IP w~tablicy adresów znajduje się indeks pamięci, w~którym przechowywane są wszystkie szczegóły.
Liczba adresów w~takiej pamięci wynosi nierzadko nawet 40 tysięcy.
Jako że adresy IP są reprezentowane przez 32 bity, tyle właśnie jest argumentów funkcji.
Dodatkowo w~wielu routerach mamy do czynienia z~dynamicznymi protokołami trasowania, które wymagają częstych zmian w~tablicy adresów.
Jest to w~związku z~tym bardzo dobry przykład zastosowania funkcji generowania indeksów.


\section{Algorytm Sasao}
Rozwiązanie problemu nadmiernej pojemności pamięci o~$n$ wejściach adresowych przy wykrywaniu $k$ wekrtorów indeksowanych ($k<<2^n$),
spośród $2^n$ możliwych wektorów wejściowych,
zaproponował Sasao \cite{sasao-workshop, sasao-recent, sasao-s-min, sasao-synthesis}.
Istotą tego rozwiązania jest
\begin{enumerate}[label=\alph*)]
\item Obliczanie minimalnej liczby argumentów potrzebnych do reprezentacji funkcji specyfikowanej $k$ wektorami $n$-bitowymi,
\item Dekompozycja liniowa funkcji obliczonej w~punkcie a)
\end{enumerate}

Dekompozycja liniowa polega na łączeniu argumentów w~grupy i~podawaniu ich na wejście dwu lub więcej argumentowych bramek XOR.
Algorytm s-Min, profesora Sasao, znajduje takie grupy.
Istnieją jednak przykłady,
dla których wskazanie żadnego takiego zbioru argumentów nie jest możliwe,
lub występuje ich mało w~stosunku do całkowitej liczby argumentów.
Z tego powodu częscią algorytmu Sasao są przekształcenia linowe,
mające na celu zwiększenie dostępnej liczby grup do dekompozycji bramkami XOR.
Dla funkcji ,,1 z~20'' (tabela numer \ref{fig:1of20}),
posiadającej $n=20$ argumentów oraz $k=20$ wierszy,
algorytm Sasao,
dla różnych wersji s-Min,
uzyskuje wynik kompresji argumentów od 14 do 8.

\begin{table}
\centering
\includegraphics[width = 13cm]{chapter01/1of20.png}
\caption{Przygład funkcji 1 z~20 (Źródło własne).}
\label{fig:1of20}
\end{table}

Z punktu widzenie syntezy funkcji generowania indeksów propozycja Sasao jest jak najbardziej prawidłowa.
Niestety zastosowany w~tej metodzie algorytm redukcji argumentów,
jak też prcedura obliczania dekompozycji liniowej nie są najskuteczniejsze.
Nowa metoda powstała na podstawie istniejących algorytmów z~dziedziny syntezy logicznej,
dla tego samego przykładu osiąga kompresję do 6 argumentów.
Tym samym pozwala wykorzysywać mniejsze i~tańsze urządzenie do rozwiązywania tych samych problemów,
albo na obsługę przoblemów, które wcześniej były zbyt zlożone,
za pomocą do tej pory używanego sprzętu komputerowego.

%Idea stojąca za takim rozwiązaniem bierze się stąd, że do rozróżnienia 1.3 miliona wektorów z~poprzedniego przykładu może wystarczyć jedynie 21 bitów.
%Gdyby udało się faktycznie zredukować liczbę wejść z~40 do 21 bitów, rozmiar pamięci zmniejszyłby się z~21 terabitów do 42 megabitów.
%\begin{multline} \\
%2^{21} \cdot \log_2 1300000 = \\
%= 2^{10} \cdot 2^{10} \cdot 2 \cdot 21 = 42 [Mbit] \\
%\end{multline}
%Zmniejszenie rozmiaru pamięci głównej nie pozwalałoby na jednoznaczne określenie czy dany wektor jest wektorem poszukiwanym.
%Wskazywałoby jedynie numer jedynego wiersza z~poszukiwanych, który ma szansę być identyczny z~wektorem sprawdzanym.
%Potrzebne jest zatem przechowanie wszystkich kompletnych wektorów poszukiwanych w~drugiej pamięci.
%W~naszym przykładzie wektorów jest 1,3 miliona i~każdy ma 40 bitów.
%Rozmiar dodatkowej pamięci wyniósłby w~takim razie 50 megabitów.


%W~zaproponowanym rozwiązaniu częścią, która ma największe znaczenie na rozmiar niezbędnej pamięci,
%jest wyznaczenie funkcji pozwalającej na jak największe zmniejszenie liczby wejść do głównej pamięci.
%\textbf{(Wymaga uzupełnienia z~artykułem)} W~pracy Profesora Sasao dla konkretnej funkcji ten wynik pozwala na ograniczenie wejść z~\textbf{N do X}.
%Sposobem zaproponowanym w~pracy uzyskujemy zmniejszenie z~\textbf{N do Y}.
%Warto zauważyć, że zmniejszenie o~jeden argument powoduje dwukrotne zmniejszenie wymaganej pamięci.



%\section{Cel pracy - do usunięcia}
%
%Celem pracy było opracowanie algorytmu generowania indeksów na podstawie istniejących algorytmów z~dziedziny syntezy logicznej
%oraz porównanie nowej metody z~wynikami aktualnych prac badawczych z~zakresu generowania indeksów.
%Wyniki obliczeń dla przykładu referencyjnego,
%przeprowadzone przed przystąpieniem do właściwych badań,
%potwierdziły konkurencyjność algorytmu opartego na połączeniu redukcji argumentów oraz dekompozycji liniowej.
%W porównaniu z~aktualnymi artykułami Profesora Sasao,
%nowa metoda pozwala na zmniejszenie zużycia zasobów potrzebnych do realizacji funkcji w~strukturach programowalnych.
%Tym samym pozwala wykorzysywać mniejsze i~tańsze urządzenie do rozwiązywania tych samych problemów,
%albo na obsługę przoblemów, które wcześniej były zbyt zlożone,
%za pomocą do tej pory używanego sprzętu komputerowego.
