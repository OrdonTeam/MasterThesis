\chapter{Wstęp}

W~dzisiejszych czasach gromadzimy ogromne ilości danych.
W~związku z~tym obserwujemy rozwój dziedzin eksploracji i~analizy danych.
Wraz z~tym rozwojem pojawia się coraz więcej problemów ściśle związanych z~koniecznością precyzyjnego wyodrębniania właściwych informacji z~ogromnej masy niepotrzebnych danych.
Szczególnym przypadkiem analizy danych są dane binarne, reprezentowane wektorami binarnymi, o~stosunkowo dużej liczbie bitów i~małej ilości informacji,
którą należy wyselekcjonować.
Jeżeli na dodatek zbiór danych selekcjonowanych podlega częstym aktualizacjom,
to strukturę sprzętową układów dokonujących wyszukiwania trzeba często zmieniać.
Rozwiązaniem problemu częstych aktualizacji sprzętowych są struktury programowalne FPGA ze~wbudowanymi pamięciami ROM.
Niestety w~przypadku takiego rozwiązania barierą może okazać się zbyt mała pojemność pamięci,
niewystarczająca do realizacji funkcji przykładowo o~40 argumentach i~10 wierszach.
Można jednak przypuszczać, że binarna tablica takich wektorów zawiera kolumny $x_i, x_j, ..., x_k$,
które ustawione obok siebie reprezentują różne liczby binarne \cite{sasao-workshop}.
Zwykle takich kolumn jest znacznie mniej od liczby zmiennych wektorów selekcjonowanych.
Zatem jedną z~metod rozwiązania zadania selekcji jest redukcja argumentów,
a cały problem jest określany mianem syntezy funkcji generowania indeksów (adresów).

Zagadnienie redukcji argumentów jest tożsame z~problemem redukcji atrybutów.
W~tym ujęciu zagadnienie to było badane bardzo intensywnie, między innymi w~pracach \cite{fast-algorithm, efektywna-procedura, new-reduction, steinbach-posthoff, skowron-rauszer, slezak, novel-method}.
Jednym z~najbardziej znanych algorytmów redukcji atrybutów jest algorytm zastosowany w~systemie RSES \cite{rses}.
Algorytm ten został usprawniony w~artykule \cite{efektywna-procedura}.
Wynikiem jego działania jest zbiór wszystkich minimalnych rozwiązań zadanego problemu redukcji atrybutów.
W~ramach niniejszej pracy stworzono realizację obliczającą wyłącznie jedno minimalne rozwiązanie (rozdział \ref{chapter:reduction}).
Analogiczne zadanie podjęto w~zespole prof. Sasao.
W~referacie \cite{sasao-workshop} zasygnalizowano,
że jednym z~najważniejszych zadań z~dziedziny wykrywania wzorców jest opracowanie szybkiego algorytmu redukcji.

Drugim aktualnym zagadnieniem syntezy logicznej w~projektowaniu generatorów adresu jest dekompozycja liniowa.
Przez dekompozycję liniową funkcji $F$ rozumie się przedstawienie jej w~postaci złożeń funkcji $H$ oraz $G_i$ ($F = H(G_1, G_2, ...,  X)$),
w~której składowymi $G_i$ dekompozycji są dwuargumentowe funkcje $XOR$:
\begin{equation}
F = H (x_i \oplus x_j, x_k \oplus x_l, ..., X)
\end{equation}
W powyższym równaniu $X$ oznacza zbiór argumentów niewykorzystany w~procesie dekompozycji.

Do najbardziej znanych algorytmów dekompozycji liniowej stosowanej w~generatorach adresów należy algorytm s-Min \cite{sasao-recent, sasao-s-min}.
W~artykule \cite{redukcja-kompresja} wykazano, że algorytm s-Min jest mało skuteczny i~zaproponowano całkowicie inne ujęcie tego zagadnienia.
W~niniejszej pracy metodykę tę przystosowano do heurystycznych obliczeń funkcji generowania indeksów.
Cechą charakterystyczną tej propozycji jest zastosowanie twierdzenia wiążącego dekompozycję liniową
z~tak zwanym zbiorem rozróżnialności [rozdział \ref{section:single-decomposition}].
W~rezultacie opracowano metodykę oraz algorytm (i jego implementację),
do heurystycznych obliczeń funkcji indeksowania, który można wykorzystać w~praktyce [rozdział \ref{chapter:decomposition}].
Przykłady takich dekompozycji podano w~rozdziale \ref{chapter:research}.

\section{Problem generowania indeksów w~technice cyfrowej}

Problem generowania indeksów (adresów) dotyczy szczególnie takiej podgrupy zagadnień czasu rzeczywistego,
w~których warunki podejmowania decyzji są zmienne w~czasie.
Sztandarowym zastosowaniem algorytmów generowania indeksów jest obsługa pakietów w~routerach IP.
Spotykamy się tam z~problemami filtrowania ruchu z~niepożądanych adresów (firewall) lub wybierania portów wyjściowych (routing).
W~obu tych przypadkach znajdują zastosowanie struktury programowalne FPGA lub EEPROM
ze względu na połączenie szybkości działania porównywalnej z~rozwiązaniami ASIC oraz elastyczności
jak w~przypadku rozwiązań programowych.
Wyzwania,
jakie są stawiane przed algorytmami generowania indeksów,
są związane z~częstymi zmianami danych.
Indeksy często muszą być generowane w~czasie rzeczywistym wraz z~aktualizacją danych.
Ponadto ze względu na dużą liczbę argumentów,
które muszą być brane pod uwagę (32 bity - adresy IPv4, 48 bitów - adresy MAC),
wymagane są optymalizacje pozwalające wykorzystywać pamięci dostępne w~strukturach programowalnych.
Przykładowo przy funkcji o~$n=6$ argumentach wszystkich wektorów binarnych jest $2^n = 2^6 = 64$.
Jeżeli spośród wszystkich wektorów poszukiwanych jest $k=6$ konkretnych, wtedy dla każdego z~64 wektorów musielibyśmy przechować 3 bity ($\log_26$).
Dla takiego przykładu minimalna potrzebna pamięć to $64 \cdot 3 = 192$ bity.
Jednak dla mechanizmu znajdowania sygnatur wirusów składającego się z~$k=1\,300\,000$ poszukiwanych wektorów o~$n=40$ argumentach \cite{wirusy}, taka pamięć miałaby rozmiar 21 terabitów.
\begin{multline} \\
2^n \cdot \log_2 k = \\
= 2^{40} \cdot \log_2 1\,300\,000 = \\
= 2^{10} \cdot 2^{10} \cdot 2^{10} \cdot 2^{10} \cdot 21 = \\
=21 \cdot 2^{10} \cdot 2^{10} \cdot 2^{10} k = \\
=21 \cdot 2^{10} \cdot 2^{10} M = \\
=21 \cdot 2^{10} G = \\
=21 [Tbit] \\
\end{multline}

\section{Zastosowania}

Powodem rosnącej popularności funkcji generowania indeksów jest możliwość stosowania ich
w rozmaitych problemach syntezy silnie nieokreślonych funkcji boolowski (dla których zachodzi zależność $k << 2^n$).
Takie zastosowania bezpośrednio pozwalają na zastosowanie ich w~procesie projektowania układów programowalnych,
ponieważ bazują one właśnie na takich funkcjach.
Jednak zastosowania funkcji generowania indeksów są dużo szersze.
Mogą one być częścią większych algorytmów wykorzystujących poszukiwanie wzorca.
Ponadto funkcje generowania indeksów znajdują zastosowanie w~dystrybucji adresów IP,
skanowaniu wirusów oraz wykrywaniu niepożądanych danych.

\subsection{Tablica trasowania}
Protokołem warstwy sieciowej w~modelu warstwowym OSI jest protokół IP.
W~związku z~tym w~Internecie powszechne jest zagadnienie znajdowania ścieżek dla pakietów IP.
Każdy węzeł sieci przechowuje informacje dotyczące wyboru trasy dla przychodzących pakietów.
Dla każdego przechowywanego adresu IP w~tablicy adresów znajduje się indeks pamięci,
w~którym przechowywane są informacje pozwalające na przekazanie pakietu do następnego węzła.
Liczba adresów w~takiej pamięci wynosi nierzadko nawet 40 tysięcy.
Jako że adresy IPv4 są reprezentowane przez 32 bity, tyle właśnie jest argumentów funkcji.
Dodatkowo w~wielu routerach mamy do czynienia z~dynamicznymi protokołami trasowania,
które wymagają częstych zmian w~tablicy adresów.
Jest to bardzo dobry przykład zastosowania funkcji generowania indeksów.

\subsection{Skanowanie wirusów}
Proces skanowania dysku w~poszukiwaniu wirusów jest bardzo wymagający.
Skanowanie nawet małych dysków domowych komputerów potrafi zajmować długie godziny.
W związku z~tym powstają nowe rozwiązania wykorzystujące implementacje sprzętowe słynące z~szybkości przewyższającej rozwiązania programowe.
Trzeba jednak mieć na uwadze, że zagrożenia ze strony złośliwego oprogramowania zmieniają się w~czasie.
Z tego powodu rozwiązania sprzętowe, o~których mowa, wykorzystują struktury programowalne.

Same algorytmy poszukiwania wirusów są mocno skupione wokół zagadnienia wyszukiwania wzorca.
Każde znane zagrożenie posiada własną unikatową sygnaturę,
która jest potem poszukiwana w~celu jego rozpoznania.
Na pewnym etapie przetwarzania do znajdowania częściowych zgodności wykorzystywane są funkcje generowania indeksów \cite{wirusy}.

\section{Algorytm Sasao}
Rozwiązanie problemu niewystarczającej puli adresów pamięci ROM przy $n$ wejściach adresowych i~wykrywaniu $k$ wektorów indeksowanych ($k<<2^n$),
spośród $2^n$ możliwych wektorów wejściowych,
zaproponował Sasao \cite{sasao-workshop, sasao-recent, sasao-s-min, sasao-synthesis}.
Istotą tego rozwiązania jest:
\begin{enumerate}[label=\alph*)]
\item Obliczanie minimalnej liczby argumentów potrzebnych do reprezentacji funkcji specyfikowanej $k$ wektorami $n$-bitowymi,
\item Dekompozycja liniowa funkcji obliczonej w~punkcie a).
\end{enumerate}

Dekompozycja liniowa polega na łączeniu argumentów w~grupy i~podawaniu ich na wejście dwu lub więcej argumentowych bramek XOR.
Algorytm s-Min, profesora Sasao, znajduje takie grupy.
Istnieją jednak przykłady,
dla których wskazanie jakiegokolwiek takiego zbioru argumentów nie jest możliwe lub występuje ich niewiele w~stosunku do całkowitej liczby argumentów.
Z~tego powodu częścią algorytmu Sasao są przekształcenia linowe,
mające na celu zwiększenie dostępnej liczby grup do dekompozycji bramkami XOR.
Dla funkcji ,,1 z~20'' (tabela numer \ref{fig:1of20}),
posiadającej $n=20$~argumentów oraz $k=20$~wierszy,
algorytm Sasao,
dla różnych wersji s-Min,
uzyskuje wynik kompresji argumentów od 14~(2-Min) do 8~(4-Min).

\begin{table}
\centering
\includegraphics[width = 13cm]{chapter01/1of20.png}
\caption{Przykład funkcji ,,1 z~20'' (źródło własne).}
\label{fig:1of20}
\end{table}

Z punktu widzenie syntezy funkcji generowania indeksów propozycja Sasao jest jak najbardziej prawidłowa.
Niestety zastosowany w~tej metodzie algorytm redukcji argumentów,
jak też procedura obliczania dekompozycji liniowej nie jest najskuteczniejsza.
Nowa metoda,
opisana w~niniejszej pracy,
powstała na podstawie istniejących algorytmów z~dziedziny syntezy logicznej,
osiąga lepszą kompresję argumentów.
Dla tego samego przykładu ,,1 z~20'' wynosi ona 6.
Tym samym algorytm pozwala wykorzystywać mniejsze i~tańsze urządzenia do rozwiązywania tech samych problemów
albo na obsługę problemów, które wcześniej były zbyt złożone,
dla do tej pory używanego sprzętu komputerowego.

%Idea stojąca za takim rozwiązaniem bierze się stąd, że do rozróżnienia 1.3 miliona wektorów z~poprzedniego przykładu może wystarczyć jedynie 21 bitów.
%Gdyby udało się faktycznie zredukować liczbę wejść z~40 do 21 bitów, rozmiar pamięci zmniejszyłby się z~21 terabitów do 42 megabitów.
%\begin{multline} \\
%2^{21} \cdot \log_2 1300000 = \\
%= 2^{10} \cdot 2^{10} \cdot 2 \cdot 21 = 42 [Mbit] \\
%\end{multline}
%Zmniejszenie rozmiaru pamięci głównej nie pozwalałoby na jednoznaczne określenie czy dany wektor jest wektorem poszukiwanym.
%Wskazywałoby jedynie numer jedynego wiersza z~poszukiwanych, który ma szansę być identyczny z~wektorem sprawdzanym.
%Potrzebne jest zatem przechowanie wszystkich kompletnych wektorów poszukiwanych w~drugiej pamięci.
%W~naszym przykładzie wektorów jest 1,3 miliona i~każdy ma 40 bitów.
%Rozmiar dodatkowej pamięci wyniósłby w~takim razie 50 megabitów.


%W~zaproponowanym rozwiązaniu częścią, która ma największe znaczenie na rozmiar niezbędnej pamięci,
%jest wyznaczenie funkcji pozwalającej na jak największe zmniejszenie liczby wejść do głównej pamięci.
%\textbf{(Wymaga uzupełnienia z~artykułem)} W~pracy Profesora Sasao dla konkretnej funkcji ten wynik pozwala na ograniczenie wejść z~\textbf{N do X}.
%Sposobem zaproponowanym w~pracy uzyskujemy zmniejszenie z~\textbf{N do Y}.
%Warto zauważyć, że zmniejszenie o~jeden argument powoduje dwukrotne zmniejszenie wymaganej pamięci.



%\section{Cel pracy - do usunięcia}
%
%Celem pracy było opracowanie algorytmu generowania indeksów na podstawie istniejących algorytmów z~dziedziny syntezy logicznej
%oraz porównanie nowej metody z~wynikami aktualnych prac badawczych z~zakresu generowania indeksów.
%Wyniki obliczeń dla przykładu referencyjnego,
%przeprowadzone przed przystąpieniem do właściwych badań,
%potwierdziły konkurencyjność algorytmu opartego na połączeniu redukcji argumentów oraz dekompozycji liniowej.
%W porównaniu z~aktualnymi artykułami Profesora Sasao,
%nowa metoda pozwala na zmniejszenie zużycia zasobów potrzebnych do realizacji funkcji w~strukturach programowalnych.
%Tym samym pozwala wykorzysywać mniejsze i~tańsze urządzenie do rozwiązywania tych samych problemów,
%albo na obsługę przoblemów, które wcześniej były zbyt zlożone,
%za pomocą do tej pory używanego sprzętu komputerowego.
