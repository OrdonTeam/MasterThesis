\chapter{Wstęp}

W dzisiejszych czasach gromadzimy ogromne ilości danych.
W związku obserwujemy rozwój dziedziny eksploracji i analizy danych
Pewna część tych problemów jest ściśle związana z koniecznością precyzyjnego wyodrębniania właściwych danych z ogromnej masy danych niepotrzebnych
Szczególnym przypadkiem analizy danych są dane binarne, reprezentowane wektorami binarnymi o stosunkowo dużej liczbie bitów,
a małej liczbie danych które trzeba wyselekcjonować.
Jeżeli na dodatek zbiór danych selekcjonowanych podlega częstym zmianom,
to strukturę sprzętową takich układów trzeba często zmieniać.
W związku z tym rozwiązaniem tego problemu są struktury FPGA z wbudowanymi pamięciami ROM.
Ale kolejną barierą w realizacji tych układów są ograniczenia w nadmiernej pojemności pamięci ROM o dużej liczbie wejść – np. 40,
co byłoby wymagane przy selekcji niewielkiej liczby wektorów (wzorców), np. 10.
Można jednak przypuszczać, że binarna tablica takich wektorów: 40 kolumn oznaczanych $x_1,…,x_{40}$ i 10 wierszy zawiera kolumny $x_i, x_j, ..., x_k$,
które ustawione obok siebie reprezentują różne liczby binarne \cite{sasao-workshop}.
Zwykle takich kolumn jest wyraźnie mniej od liczby zmiennych (bitów) wektorów selekcjonowanych.
Zatem jedną z metod rozwiązania zadania selekcji jest redukcja argumentów,
a cały problem jest określany mianem syntezy funkcji generowania indeksów.

Zagadnienie redukcji argumentów jest tożsame z problemem redukcji atrybutów.
W tym ujęciu zagadnienie to było badane bardzo intensywnie \cite{fast-algorithm, efektywna-procedura, new-reduction, steinbach-posthoff, skowron-rauszer, slezak, novel-method}.
Jednym z najbardziej znanych algorytmów redukcji atrybutów jest algorytm zastosowany w systemie RSES \cite{rses}.
Algorytm ten został skutecznie usprawniony w pracy inżynierskiej \cite{efektywna-procedura},
natomiast w ramach niniejszej pracy stworzono realizację obliczającą jedno rozwiązanie [rozdz. ??].
Analogiczne zadanie podjęto w zespole prof. Sasao,
a w referacie \cite{sasao-workshop} znaczenie redukcji zasygnalizowano informacją,
iż opracowanie algorytmu redukcji jest jednym z najważniejszych zadań.

Drugim aktualnym zagadnieniem syntezy logicznej w projektowaniu generatorów adresu jest dekompozycja liniowa.
Przez dekompozycję liniową rozumie się dekompozycję $F = H(G1, G2, ...,  X)$,
w której składowymi dekompozycji G są dwuargumentowe funkcje $EXOR$:
\begin{equation}
F = H (x_i \oplus x_j, x_k \oplus x_l, ..., X)
\end{equation}
Do najbardziej znanych algorytmów dekompozycji liniowej stosowanej w generatorach adresów należy algorytm s-min \cite{sasao-recent, sasao-s-min}.
W artykule \cite{redukcja-kompresja} wykazano, że algorytm s-min jest mało skuteczny i zaproponowano całkowicie inne ujęcie tego zagadnienia.
W niniejszej pracy metodykę tę przystosowano do heurystycznych obliczeń funkcji generowania indeksów.
Cechą charakterystyczną tej propozycji jest zastosowanie twierdzenia wiążącego dekompozycję liniową z tzw. zbiorem rozróżnialności [rozdz. ??].
W rezultacie opracowano metodykę oraz algorytm (i jego implementację) do heurystycznych obliczeń praktycznych funkcji indeksowania [rozdz. …].
Przykłady takich dekompozycji podano w rozdz. ??.

\section{Problem generowania indeksów w technice cyfrowej}

Problem generowania indeksów (adresu) dotyczy szczególnie tej podgrupy problemów czasu rzeczywistego,
w których warunki podejmowania decyzji są zmienne w czasie.
Sztandarowym zastosowaniem algorytmów generowania indeksów jest obsługa pakietów w routerach IP.
Spotykamy się tam z problemami filtrowania ruchu z niepożądanych adresów (firewall) lub wybierania portów wyjściowych (routing).
W obu przypadkach znajdują zastosowanie struktury programowalne FPGA lub EEPROM
ze względu na połączenie szybkości działania porównywalnej z rozwiązaniami ASIC oraz elastyczności jak w przypadku rozwiązań programowych.
Wyzwania jakie są stawiane przed algorytmami generowania indeksów są związane częstymi zmianami danych.
Indeksy często muszą być generowane w czasie rzeczywistym wraz ze zmianami danych.
Ponadto ze względu na dużą liczbę argumentów,
które muszą być brane pod uwagę (32 bity - adresy IPv4, 48 bitów - adresy mac),
wymagane są optymalizacje pozwalające wykorzystywać pamięci dostępne w strukturach programowalnych.

\section{Cel pracy - do usunięcia}

Celem pracy było opracowanie algorytmu generowania indeksów na podstawie istniejących algorytmów z dziedziny syntezy logicznej
oraz porównanie nowej metody z wynikami aktualnych prac badawczych z zakresu generowania indeksów.
Wyniki obliczeń dla przykładu referencyjnego,
przeprowadzone przed przystąpieniem do właściwych badań,
potwierdziły konkurencyjność algorytmu opartego na połączeniu redukcji argumentów oraz dekompozycji liniowej.
W porównaniu z aktualnymi artykułami Profesora Sasao,
nowa metoda pozwala na zmniejszenie zużycia zasobów potrzebnych do realizacji funkcji w strukturach programowalnych.
Tym samym pozwala wykorzysywać mniejsze i tańsze urządzenie do rozwiązywania tych samych problemów,
albo na obsługę przoblemów, które wcześniej były zbyt zlożone,
za pomocą do tej pory używanego sprzętu komputerowego.
