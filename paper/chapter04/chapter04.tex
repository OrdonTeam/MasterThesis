\chapter{Badania}
\label{chapter:research}

W rozdziale tym zostały zaprezentowane wyniki badań przeprowadzone dla przykładowych funkcji referencyjnych.
Wszystkie funkcjie opisane w tym rozdziale znajdują się w załącznikach niniejszej pracy dyplomowej.

\section{Filtr f5}

Przykładem potwierdzającym skuteczność algorytmu opracowanego w ramach niniejszej pracy
może być układ arytmetyki rozproszonej [2] filtru f5 [3].
Reprezentacja tego układu za pomocą w pełni określonych funkcji boolowskich
jest opisana tablicą o 11 zmiennych wejściowych i 11 wyjściach.
Implementacja tego układu [7] w strukturze Virtex-7 wykonana za pomocą programu Vivado 2015.4.2
wymaga zastosowania 531 komórek (330 6-wejściowych, 86 5-wejściowych; pozostałe 4, 3 i 2-wejściowe).
Opracowany algorytm redukcji i kompresji argumentów (dalej nazywany algorytmem RedKomp)
redukuje tę funkcję do 7 argumentów,
umożliwiając jej realizację
(wykonywana w tej samej strukturze programem Vivado)
w strukturze zajmującej 9 komórek 6-wejściowych oraz po jednej 5, 4 i 2-wejściowej.

