\newpage
\vspace{10cm}

\newpage
\begin{center}
	\textbf{Synteza funkcji generowania indeksów metodą redukcji i~kompresji argumentów}
\end{center}
\noindent{\textbf{Streszczenie}}

Funkcje boolowskie można realizować w~ukłądach programowalnych przy użyciu pamięci adresowanych.
Problemem, który się pojawia, jest rozmiar tych pamięci.
Zaprezentowany w~niniejszej pracy algorytm generowania funkcji indeksów skutecznie ogranicza ten problem.
Korzysta on z~faktu,
że wiele funkcji boolowskich o~$n$ argumentach i~$k$ wierszach może być reprezentowana przy pomocy mniejszej liczby argumentów,
jeżeli $k<<2^n$.
Proponowane rozwiązanie,
korzystające z~istniejących algorytmów syntezy logicznej,
zmniejsza liczbę argumentów wymaganych do reprezentacji funkcji.

\textit{\textbf{Słowa kluczowe:}} funkcja generowania indeksów; dekompozycja liniowa; redukcja atrybutów; eksploracja danych; minimalne pokrycie kolumnowe.

	\vspace{1cm}

\begin{center}
    \textbf{Index generation function based on argument reduction and linear function decomposition}
\end{center}
\noindent{\textbf{Summary}}

Boolean functions can be realised in field-programmable utilising address memory.
However for large functions memory size can be insufficient.
The thesis provides an index generation function algorithm to address this issue.
Many boolean functions with $n$ arguments and a~weight of $k$ can be represented using fewer than $n$ arguments if only $k<<2^n$ condition occurs.
Suggested solution uses existing logic synthesis algorithms to reduce set of attributes required in function realisation.

\textit{\textbf{Keywords:}} index generation function; linear decomposition; argument reduction; data exploration; unate covering.

	\vspace*{\stretch{1}}
