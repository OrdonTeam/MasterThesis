\newpage
\vspace{10cm}

\newpage
\begin{center}
	\textbf{Synteza funkcji generowania indeksów metodą redukcji i~kompresji argumentów}
\end{center}
\noindent{\textbf{Streszczenie}}

Funkcje boolowskie można realizować w~układach programowalnych przy użyciu pamięci ROM.
Niestety w~przypadku funkcji o~dużej liczbie argumentów pamięci te mają niewystarczającą pulę adresów.
Zaprezentowany w~niniejszej pracy algorytm generowania funkcji indeksów skutecznie ogranicza ten problem.
Algorytm korzysta z~faktu,
że wiele funkcji boolowskich o~$n$ argumentach i~$k$ wierszach może być reprezentowana przy pomocy mniejszej liczby argumentów,
jeżeli $k<<2^n$.
Proponowane rozwiązanie,
oparte na istniejących metodach syntezy logicznej,
zmniejsza liczbę argumentów wymaganych do reprezentacji funkcji.

\textit{\textbf{Słowa kluczowe:}} funkcja generowania indeksów, dekompozycja liniowa, redukcja atrybutów, eksploracja danych, minimalne pokrycie kolumnowe.

	\vspace{1cm}

\begin{center}
    \textbf{Index generation function based on argument reduction and linear function decomposition}
\end{center}
\noindent{\textbf{Summary}}

Boolean functions can be implemented in a~field-programmable utilising ROM memory.
However for functions with many arguments this memory can have insufficient number of addresses.
The thesis provides an index generation function algorithm to solve this issue.
Many boolean functions with $n$ arguments and $k$ rows can be represented using fewer than $n$ arguments if only $k<<2^n$ condition occurs.
Suggested solution uses existing logic synthesis algorithms to reduce the set of arguments required in the function implementation.

\textit{\textbf{Keywords:}} index generation function, linear decomposition, argument reduction, data exploration, unate covering.

	\vspace*{\stretch{1}}
\cleardoublepage
