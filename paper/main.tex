
\linespread{1.3}								% 1.3 do interlinii 1.5


% w�asne pakiety

%===============================================================================
% Ustawienia dokumentu

\frenchspacing

% ustawienia wymiar�w


\oddsidemargin 20mm							% margines nieparzystych stron
\evensidemargin 20mm							% margines parzystych stron
%\headheight 15pt								% wysoko�� paginy g�rnej
%\topmargin 20mm									% margines g�rny

% styl paginacji
\pagestyle{fancy}
%\renewcommand{\chaptermark}[1]{\markboth{#1}{}}
%\renewcommand{\sectionmark}[1]{\markright{\thesection\ #1}{}}

% nag��wek 
\fancyhf{}
\fancyfoot[LE,RO]{\thepage}
\fancyhead[C]{\small\nouppercase{\rightmark}}

%\fancyhead[R]{\small\nouppercase{\leftmark}}
\renewcommand{\headrulewidth}{0.1pt}
\renewcommand{\footrulewidth}{0pt}

% nag��wek w stylu plain 
\fancypagestyle{plain}
{
\fancyhf{}
\renewcommand{\headrulewidth}{0pt}
\renewcommand{\footrulewidth}{0pt}
}

% ta sekwencja tworzy czyste kartki na stronach po \cleardoublepage
\makeatletter
\def\cleardoublepage{\clearpage\if@twoside \ifodd\c@page\else
	\hbox{}
	\vspace*{\fill}
	\thispagestyle{empty}
	\newpage
	\if@twocolumn\hbox{}\newpage\fi\fi\fi}
\makeatother

%===============================================================================
% Zmienne �rodowiskowe i polecenia

% definicja
\newtheorem{definition}{Definicja}[chapter]

% twierdzenie
\newtheorem{theorem}{Twierdzenie}[chapter]

% obcoj�zyczne nazwy
\newcommand{\foreign}[1]{\emph{#1}}

% pozioma linia
\newcommand{\horline}{\noindent\rule{\textwidth}{0.4mm}}

% wstawianie obrazk�w {plik}{caption}{opis}


%===============================================================================
% ustawienia pakietu hyperref

\hypersetup{
    colorlinks,
    citecolor=black,
    filecolor=black,
    linkcolor=black,
    urlcolor=blue
}
\hypersetup
{
%colorlinks=true,			% false: boxed links; true: colored links
%linkcolor=black,			% color of internal links
%citecolor=black,			% color of links to bibliography
%filecolor=black,			% color of file links
%urlcolor=black			% color of external links
}
			% do³¹czanie plików stylu
\usepackage{pwtitle}
\usepackage{multirow}
\usepackage{tocloft}
\usepackage{chngcntr}
\usepackage{varwidth}
\usepackage{indentfirst}
\usepackage{polski}
\usepackage{float}
\usepackage[utf8]{inputenc}
\usepackage{listings}
\usepackage{slashbox}
\usepackage[table]{xcolor}
\usepackage{graphicx,pdflscape}
\usepackage{placeins}
\usepackage[final]{pdfpages}
\usepackage{longtable}

\input{listing.tex}
\counterwithout{figure}{chapter}
\counterwithout{equation}{chapter}
\lstset{
  numberbychapter=false,
	extendedchars=true
}
%\counterwithout{lstlisting}{chapter}

\usepackage{titlesec}

\titleformat{\chapter}{\sffamily\huge\bf}{\thechapter.}{20pt}{\huge\bf}
%\usepackage{geometry}
\begin{document}
\hyphenation{Par-tial-ac-ti-vi-ty}
\hyphenation{zwra-ca}
\hyphenation{get-Ho-ri-zon-tal-View-Angle}

%===============================================================================
% Front

\frontmatter

\frontmatter

%\frontmatter

%\frontmatter

%\input{front.tex}
\input{front//biography.tex}
\input{front//mark.tex}
\input{front//summary.tex}
\input{front//statement.tex}
\input{front//dedication.tex}
{\tableofcontents}

	\newpage
	\thispagestyle{empty}

\begin{flushright}

  \begin{varwidth}[t]{\textwidth}
	Kierunek studiów: Telekomunikacja\\
	Specjalność: Telekomunikacja \\
	Numer albumu: 236501\\
	Data urodzenia: 04.02.1991 \\
	Data rozpoczęcia studiów: 01.10.2014 \\
  \end{varwidth}

\end{flushright}
	\vspace*{\stretch{1}}

\begin{center}
    \textbf{\textbf{Życiorys}}
\end{center}

	\vspace{0.5cm}

Urodziłem się 4 lutego 1991 roku w Olsztynie.
W roku 2010 uzyskałem tytuł finalisty LXI Olimpiady Matematycznej oraz
ukończyłem II Liceum Ogólnokształcące im. Konstantego Ildefonsa Gałczyńskiego w Olsztynie.
W tym samym roku rozpocząłem studia na Politechnice Warszawskiej na kierunku Elektronika, Informatyka i Telekomunikacja.
Od 2011 roku współpracowałem z prof. T. Łubą oraz dr inż. G. Borowikiem w ramach koła Eksploracji Danych.
W czasie studiów dwukrotnie zdobyłem stypendium rektora za wyniki w nauce.
Przez rok na przełomie 2013 i 2014 roku pracowałem jako Java Developer w firmie IMPAQ.
Po ukończeniu stódiów inżynierskich w październiku 2014 rozpocząłem studia drugiego stopnia.
Od marca 2015 łączyłem studia z pracą na stanowisku Android Developer w firmie El Passion.

	\vspace{1cm}

\begin{flushright}
	\begin{minipage}{5cm}
		\dotfill \\[-0.7cm]
		\begin{center}
		\small Podpis
		\end{center}
	\end{minipage}
\end{flushright}

	\vspace{2cm}

\newpage
	\thispagestyle{empty}
\begin{flushleft}
    	\begin{minipage}{11cm}
		Egzamin dyplomowy: \\
		Złożył egzamin dyplomowy w dniu: \dotfill \\
		z wynikiem: \dotfill \\
		Ogólny wynik studiów: \dotfill \\
		Dodatkowe uwagi i wnioski Komisji: \dotfill \\
		\dotfill
	\end{minipage}
\end{flushleft}
\newpage
\vspace{10cm}

\newpage
\begin{center}
	\textbf{Synteza funkcji generowania indeksów metodą redukcji i kompresji argumentów}
\end{center}
\noindent{\textbf{Streszczenie}}

Funkcje boolowskie można realizować w~ukłądach programowalnych przy użyciu pamięci adresowanych.
Problemem, który się pojawia, jest rozmiar tych pamięci.
Zaprezentowany w~niniejszej pracy algorytm generowania funkcji indeksów skutecznie ogranicza ten problem.
Korzysta on z faktu,
że wiele funkcji boolowskich o~$n$ argumentach i $k$ wierszach może być reprezentowana przy pomocy mniejszej liczby argumentów,
jeżeli $k<<2^n$.
Proponowane rozwiązanie,
korzystające z istniejących algorytmów syntezy logicznej,
zmniejsza liczbę argumentów wymaganych do reprezentacji funkcji.

\textit{\textbf{Słowa kluczowe:}} funkcja generowania indeksów; dekompozycja liniowa; redukcja atrybutów; eksploracja danych; minimalne pokrycie kolumnowe.

	\vspace{1cm}

\begin{center}
    \textbf{Index generation function based on argument reduction and linear function decomposition}
\end{center}
\noindent{\textbf{Summary}}

Boolean functions can be realised in field-programmable utilising address memory.
However for large functions memory size can be insufficient.
The thesis provides an index generation function algorithm to address this issue.
Many boolean functions with $n$ arguments and a~weight of $k$ can be represented using fewer than $n$ arguments if only $k<<2^n$ condition occurs.
Suggested solution uses existing logic synthesis algorithms to reduce set of attributes required in function realisation.

\textit{\textbf{Keywords:}} index generation function; linear decomposition; argument reduction; data exploration; unate covering.

	\vspace*{\stretch{1}}

\newpage
\thispagestyle{empty}

\vspace*{\stretch{1}}

\begin{center}
\LARGE\textsc{Oświadczenie}
\end{center}

\vspace{1cm}

Oświadczam, że Pracę Dyplomową pod tytułem \emph{,,Synteza funkcji generowania indeksów metodą redukcji i~kompresji argumentów''}, którą kierował prof. dr hab. inż. Tadeusz Łuba, wykonałem samodzielnie, co poświadczam własnoręcznym podpisem.

\vspace{2cm}

\begin{flushright}
\begin{minipage}{5cm}
	\dotfill \\[-0.7cm]
	\begin{center}
	\small{Karol Kowalski}
	\end{center}
\end{minipage}
\end{flushright}

\vspace*{\stretch{1}}

\newpage
\thispagestyle{empty}
\vspace*{\stretch{1}}
\begin{flushright}
\textit{Składam serdeczne podziękowania
panu prof. dr hab. inż. Tadeuszowi Łubie
za pomoc w trakcie przygotowywania pracy.}
\end{flushright}
{\tableofcontents}

	\newpage
	\thispagestyle{empty}

\begin{flushright}

  \begin{varwidth}[t]{\textwidth}
	Kierunek studiów: Telekomunikacja\\
	Specjalność: Telekomunikacja \\
	Numer albumu: 236501\\
	Data urodzenia: 04.02.1991 \\
	Data rozpoczęcia studiów: 01.10.2014 \\
  \end{varwidth}

\end{flushright}
	\vspace*{\stretch{1}}

\begin{center}
    \textbf{\textbf{Życiorys}}
\end{center}

	\vspace{0.5cm}

Urodziłem się 4 lutego 1991 roku w Olsztynie.
W roku 2010 uzyskałem tytuł finalisty LXI Olimpiady Matematycznej oraz
ukończyłem II Liceum Ogólnokształcące im. Konstantego Ildefonsa Gałczyńskiego w Olsztynie.
W tym samym roku rozpocząłem studia na Politechnice Warszawskiej na kierunku Elektronika, Informatyka i Telekomunikacja.
Od 2011 roku współpracowałem z prof. T. Łubą oraz dr inż. G. Borowikiem w ramach koła Eksploracji Danych.
W czasie studiów dwukrotnie zdobyłem stypendium rektora za wyniki w nauce.
Przez rok na przełomie 2013 i 2014 roku pracowałem jako Java Developer w firmie IMPAQ.
Po ukończeniu stódiów inżynierskich w październiku 2014 rozpocząłem studia drugiego stopnia.
Od marca 2015 łączyłem studia z pracą na stanowisku Android Developer w firmie El Passion.

	\vspace{1cm}

\begin{flushright}
	\begin{minipage}{5cm}
		\dotfill \\[-0.7cm]
		\begin{center}
		\small Podpis
		\end{center}
	\end{minipage}
\end{flushright}

	\vspace{2cm}

\newpage
	\thispagestyle{empty}
\begin{flushleft}
    	\begin{minipage}{11cm}
		Egzamin dyplomowy: \\
		Złożył egzamin dyplomowy w dniu: \dotfill \\
		z wynikiem: \dotfill \\
		Ogólny wynik studiów: \dotfill \\
		Dodatkowe uwagi i wnioski Komisji: \dotfill \\
		\dotfill
	\end{minipage}
\end{flushleft}
\newpage
\vspace{10cm}

\newpage
\begin{center}
	\textbf{Synteza funkcji generowania indeksów metodą redukcji i kompresji argumentów}
\end{center}
\noindent{\textbf{Streszczenie}}

Funkcje boolowskie można realizować w~ukłądach programowalnych przy użyciu pamięci adresowanych.
Problemem, który się pojawia, jest rozmiar tych pamięci.
Zaprezentowany w~niniejszej pracy algorytm generowania funkcji indeksów skutecznie ogranicza ten problem.
Korzysta on z faktu,
że wiele funkcji boolowskich o~$n$ argumentach i $k$ wierszach może być reprezentowana przy pomocy mniejszej liczby argumentów,
jeżeli $k<<2^n$.
Proponowane rozwiązanie,
korzystające z istniejących algorytmów syntezy logicznej,
zmniejsza liczbę argumentów wymaganych do reprezentacji funkcji.

\textit{\textbf{Słowa kluczowe:}} funkcja generowania indeksów; dekompozycja liniowa; redukcja atrybutów; eksploracja danych; minimalne pokrycie kolumnowe.

	\vspace{1cm}

\begin{center}
    \textbf{Index generation function based on argument reduction and linear function decomposition}
\end{center}
\noindent{\textbf{Summary}}

Boolean functions can be realised in field-programmable utilising address memory.
However for large functions memory size can be insufficient.
The thesis provides an index generation function algorithm to address this issue.
Many boolean functions with $n$ arguments and a~weight of $k$ can be represented using fewer than $n$ arguments if only $k<<2^n$ condition occurs.
Suggested solution uses existing logic synthesis algorithms to reduce set of attributes required in function realisation.

\textit{\textbf{Keywords:}} index generation function; linear decomposition; argument reduction; data exploration; unate covering.

	\vspace*{\stretch{1}}

\newpage
\thispagestyle{empty}

\vspace*{\stretch{1}}

\begin{center}
\LARGE\textsc{Oświadczenie}
\end{center}

\vspace{1cm}

Oświadczam, że Pracę Dyplomową pod tytułem \emph{,,Synteza funkcji generowania indeksów metodą redukcji i~kompresji argumentów''}, którą kierował prof. dr hab. inż. Tadeusz Łuba, wykonałem samodzielnie, co poświadczam własnoręcznym podpisem.

\vspace{2cm}

\begin{flushright}
\begin{minipage}{5cm}
	\dotfill \\[-0.7cm]
	\begin{center}
	\small{Karol Kowalski}
	\end{center}
\end{minipage}
\end{flushright}

\vspace*{\stretch{1}}

\newpage
\thispagestyle{empty}
\vspace*{\stretch{1}}
\begin{flushright}
\textit{Składam serdeczne podziękowania
panu prof. dr hab. inż. Tadeuszowi Łubie
za pomoc w trakcie przygotowywania pracy.}
\end{flushright}
{\tableofcontents}
			% definicja strony tytu³owej

%===============================================================================
% Rozdzia³y - dla porz¹dku pliki dla ka¿dego z rozdzia³ów znajduj± siê w oddzielnych katalogach
\mainmatter
\newgeometry{inner=40mm, outer=20mm}

\chapter{Wstęp}

W dzisiejszych czasach gromadzimy ogromne ilości danych.
W związku obserwujemy rozwój dziedziny eksploracji i~analizy danych
Pewna część tych problemów jest ściśle związana z~koniecznością precyzyjnego wyodrębniania właściwych danych z~ogromnej masy danych niepotrzebnych
Szczególnym przypadkiem analizy danych są dane binarne, reprezentowane wektorami binarnymi o~stosunkowo dużej liczbie bitów,
a małej liczbie danych które trzeba wyselekcjonować.
Jeżeli na dodatek zbiór danych selekcjonowanych podlega częstym zmianom,
to strukturę sprzętową takich układów trzeba często zmieniać.
W związku z~tym rozwiązaniem tego problemu są struktury FPGA z~wbudowanymi pamięciami ROM.
Ale kolejną barierą w~realizacji tych układów są ograniczenia w~nadmiernej pojemności pamięci ROM o~dużej liczbie wejść – np. 40,
co byłoby wymagane przy selekcji niewielkiej liczby wektorów (wzorców), np. 10.
Można jednak przypuszczać, że binarna tablica takich wektorów: 40 kolumn oznaczanych $x_1,…,x_{40}$ i~10 wierszy zawiera kolumny $x_i, x_j, ..., x_k$,
które ustawione obok siebie reprezentują różne liczby binarne \cite{sasao-workshop}.
Zwykle takich kolumn jest wyraźnie mniej od liczby zmiennych (bitów) wektorów selekcjonowanych.
Zatem jedną z~metod rozwiązania zadania selekcji jest redukcja argumentów,
a cały problem jest określany mianem syntezy funkcji generowania indeksów.

Zagadnienie redukcji argumentów jest tożsame z~problemem redukcji atrybutów.
W tym ujęciu zagadnienie to było badane bardzo intensywnie \cite{fast-algorithm, efektywna-procedura, new-reduction, steinbach-posthoff, skowron-rauszer, slezak, novel-method}.
Jednym z~najbardziej znanych algorytmów redukcji atrybutów jest algorytm zastosowany w~systemie RSES \cite{rses}.
Algorytm ten został skutecznie usprawniony w~pracy inżynierskiej \cite{efektywna-procedura},
natomiast w~ramach niniejszej pracy stworzono realizację obliczającą jedno rozwiązanie [rozdz. ??].
Analogiczne zadanie podjęto w~zespole prof. Sasao,
a w~referacie \cite{sasao-workshop} znaczenie redukcji zasygnalizowano informacją,
iż opracowanie algorytmu redukcji jest jednym z~najważniejszych zadań.

Drugim aktualnym zagadnieniem syntezy logicznej w~projektowaniu generatorów adresu jest dekompozycja liniowa.
Przez dekompozycję liniową rozumie się dekompozycję $F = H(G1, G2, ...,  X)$,
w której składowymi dekompozycji G są dwuargumentowe funkcje $EXOR$:
\begin{equation}
F = H (x_i \oplus x_j, x_k \oplus x_l, ..., X)
\end{equation}
Do najbardziej znanych algorytmów dekompozycji liniowej stosowanej w~generatorach adresów należy algorytm s-min \cite{sasao-recent, sasao-s-min}.
W artykule \cite{redukcja-kompresja} wykazano, że algorytm s-min jest mało skuteczny i~zaproponowano całkowicie inne ujęcie tego zagadnienia.
W niniejszej pracy metodykę tę przystosowano do heurystycznych obliczeń funkcji generowania indeksów.
Cechą charakterystyczną tej propozycji jest zastosowanie twierdzenia wiążącego dekompozycję liniową z~tzw. zbiorem rozróżnialności [rozdz. ??].
W rezultacie opracowano metodykę oraz algorytm (i jego implementację) do heurystycznych obliczeń praktycznych funkcji indeksowania [rozdz. …].
Przykłady takich dekompozycji podano w~rozdz. ??.

\section{Problem generowania indeksów w~technice cyfrowej}

Problem generowania indeksów (adresu) dotyczy szczególnie tej podgrupy problemów czasu rzeczywistego,
w których warunki podejmowania decyzji są zmienne w~czasie.
Sztandarowym zastosowaniem algorytmów generowania indeksów jest obsługa pakietów w~routerach IP.
Spotykamy się tam z~problemami filtrowania ruchu z~niepożądanych adresów (firewall) lub wybierania portów wyjściowych (routing).
W obu przypadkach znajdują zastosowanie struktury programowalne FPGA lub EEPROM
ze względu na połączenie szybkości działania porównywalnej z~rozwiązaniami ASIC oraz elastyczności jak w~przypadku rozwiązań programowych.
Wyzwania,
jakie są stawiane przed algorytmami generowania indeksów,
są związane częstymi zmianami danych.Indeksy często muszą być generowane w~czasie rzeczywistym wraz ze zmianami danych.
Ponadto ze względu na dużą liczbę argumentów,
które muszą być brane pod uwagę (32 bity - adresy IPv4, 48 bitów - adresy mac),
wymagane są optymalizacje pozwalające wykorzystywać pamięci dostępne w~strukturach programowalnych.
Przykładowo przy funkcji o~$n=6$ argumentach wszystkich wektorów binarnych jest 64 ($2^6$).
Jeżeli spośród wszystkich wektorów poszukiwanych jest $k=6$ konkretnych, wtedy dla każdego z~64 wektorów musielibyśmy przechować 3 bity ($\log_26$).
Dla takiego przykładu minimalna potrzebna pamięć to 192 bity.
Jednak dla mechanizmu znajdowania sygnatur wirusów składającego się z~$k=1\,300\,000$ poszukiwanych wektorów o~$n=40$ argumentach, taka pamięć miałaby rozmiar 21 terabitów.
\begin{multline} \\
2^n \cdot \log_2 k = \\
= 2^{40} \cdot \log_2 1\,300\,000 = \\
= 2^{10} \cdot 2^{10} \cdot 2^{10} \cdot 2^{10} \cdot 21 = \\
=21 \cdot 2^{10} \cdot 2^{10} \cdot 2^{10} k = \\
=21 \cdot 2^{10} \cdot 2^{10} M = \\
=21 \cdot 2^{10} G = \\
=21 [Tbit] \\
\end{multline}


\section{Zastosowania}
- Tablica trasowania
- Terminal access controller
- Memory patch circuit
- Skanowanie wirusów
- Dystrybucja adresów IP
- Skanowanie wirusów
- Wykrywanie niepożądanych danych
- Konwersja kodów

\subsection{Tablica trasowania}
W Internecie powszechne jest zagadnienie znajdowania ścieżek dla pakietów IP.
Każdy węzeł sieci przechowuje informacje dotyczące wyboru trasy dla przychodzących pakietów.
Dla każdego przechowywanego adresu IP w~tablicy adresów znajduje się indeks pamięci, w~którym przechowywane są wszystkie szczegóły.
Liczba adresów w~takiej pamięci wynosi nierzadko nawet 40 tysięcy.
Jako że adresy IP są reprezentowane przez 32 bity, tyle właśnie jest argumentów funkcji.
Dodatkowo w~wielu routerach mamy do czynienia z~dynamicznymi protokołami trasowania, które wymagają częstych zmian w~tablicy adresów.
Jest to w~związku z~tym bardzo dobry przykład zastosowania funkcji generowania indeksów.


\section{Algorytm Sasao}
Rozwiązanie problemu nadmiernej pojemności pamięci o~$n$ wejściach adresowych przy wykrywaniu $k$ wekrtorów indeksowanych ($k<<2^n$),
spośród $2^n$ możliwych wektorów wejściowych,
zaproponował Sasao \cite{sasao-workshop, sasao-recent, sasao-s-min, sasao-synthesis}.
Istotą tego rozwiązania jest
\begin{enumerate}[label=\alph*)]
\item Obliczanie minimalnej liczby argumentów potrzebnych do reprezentacji funkcji specyfikowanej $k$ wektorami $n$-bitowymi,
\item Dekompozycja liniowa funkcji obliczonej w~punkcie a)
\end{enumerate}

Dekompozycja liniowa polega na łączeniu argumentów w~grupy i~podawaniu ich na wejście dwu lub więcej argumentowych bramek XOR.
Algorytm s-Min, profesora Sasao, znajduje takie grupy.
Istnieją jednak przykłady,
dla których wskazanie żadnego takiego zbioru argumentów nie jest możliwe,
lub występuje ich mało w~stosunku do całkowitej liczby argumentów.
Z tego powodu częscią algorytmu Sasao są przekształcenia linowe,
mające na celu zwiększenie dostępnej liczby grup do dekompozycji bramkami XOR.
Dla funkcji ,,1 z~20'' (tabela numer \ref{fig:1of20}),
posiadającej $n=20$ argumentów oraz $k=20$ wierszy,
algorytm Sasao,
dla różnych wersji s-Min,
uzyskuje wynik kompresji argumentów od 14 do 8.

\begin{table}
\centering
\includegraphics[width = 13cm]{chapter01/1of20.png}
\caption{Przygład funkcji 1 z~20 (Źródło własne).}
\label{fig:1of20}
\end{table}

Z punktu widzenie syntezy funkcji generowania indeksów propozycja Sasao jest jak najbardziej prawidłowa.
Niestety zastosowany w~tej metodzie algorytm redukcji argumentów,
jak też prcedura obliczania dekompozycji liniowej nie są najskuteczniejsze.
Nowa metoda powstała na podstawie istniejących algorytmów z~dziedziny syntezy logicznej,
dla tego samego przykładu osiąga kompresję do 6 argumentów.
Tym samym pozwala wykorzysywać mniejsze i~tańsze urządzenie do rozwiązywania tych samych problemów,
albo na obsługę przoblemów, które wcześniej były zbyt zlożone,
za pomocą do tej pory używanego sprzętu komputerowego.

%Idea stojąca za takim rozwiązaniem bierze się stąd, że do rozróżnienia 1.3 miliona wektorów z~poprzedniego przykładu może wystarczyć jedynie 21 bitów.
%Gdyby udało się faktycznie zredukować liczbę wejść z~40 do 21 bitów, rozmiar pamięci zmniejszyłby się z~21 terabitów do 42 megabitów.
%\begin{multline} \\
%2^{21} \cdot \log_2 1300000 = \\
%= 2^{10} \cdot 2^{10} \cdot 2 \cdot 21 = 42 [Mbit] \\
%\end{multline}
%Zmniejszenie rozmiaru pamięci głównej nie pozwalałoby na jednoznaczne określenie czy dany wektor jest wektorem poszukiwanym.
%Wskazywałoby jedynie numer jedynego wiersza z~poszukiwanych, który ma szansę być identyczny z~wektorem sprawdzanym.
%Potrzebne jest zatem przechowanie wszystkich kompletnych wektorów poszukiwanych w~drugiej pamięci.
%W naszym przykładzie wektorów jest 1,3 miliona i~każdy ma 40 bitów.
%Rozmiar dodatkowej pamięci wyniósłby w~takim razie 50 megabitów.


%W zaproponowanym rozwiązaniu częścią, która ma największe znaczenie na rozmiar niezbędnej pamięci,
%jest wyznaczenie funkcji pozwalającej na jak największe zmniejszenie liczby wejść do głównej pamięci.
%\textbf{(Wymaga uzupełnienia z~artykułem)} W~pracy Profesora Sasao dla konkretnej funkcji ten wynik pozwala na ograniczenie wejść z~\textbf{N do X}.
%Sposobem zaproponowanym w~pracy uzyskujemy zmniejszenie z~\textbf{N do Y}.
%Warto zauważyć, że zmniejszenie o~jeden argument powoduje dwukrotne zmniejszenie wymaganej pamięci.



%\section{Cel pracy - do usunięcia}
%
%Celem pracy było opracowanie algorytmu generowania indeksów na podstawie istniejących algorytmów z~dziedziny syntezy logicznej
%oraz porównanie nowej metody z~wynikami aktualnych prac badawczych z~zakresu generowania indeksów.
%Wyniki obliczeń dla przykładu referencyjnego,
%przeprowadzone przed przystąpieniem do właściwych badań,
%potwierdziły konkurencyjność algorytmu opartego na połączeniu redukcji argumentów oraz dekompozycji liniowej.
%W porównaniu z~aktualnymi artykułami Profesora Sasao,
%nowa metoda pozwala na zmniejszenie zużycia zasobów potrzebnych do realizacji funkcji w~strukturach programowalnych.
%Tym samym pozwala wykorzysywać mniejsze i~tańsze urządzenie do rozwiązywania tych samych problemów,
%albo na obsługę przoblemów, które wcześniej były zbyt zlożone,
%za pomocą do tej pory używanego sprzętu komputerowego.



\backmatter
\pagestyle{empty}
%*******************************************************************************
% Bibliografia - spis literatury wykorzystanej przy tworzeniu pracy
%*******************************************************************************

\begin{thebibliography}{99}
\addcontentsline{toc}{chapter}{Bibliografia}
\bibliographystyle{tran}

%[1] Borowik G., Łuba T.: Fast Algorithm of Attribute Reduction Based on the complementation of Boolean Function, ch. 2, pp. 25-41, Springer International Publishing, 2014.
\bibitem{fast-algorithm} G. Borowik, T. Łuba \emph{,,Fast Algorithm of Attribute Reduction Based on the complementation of Boolean Function''}, ch. 2, pp. 25-41, Springer International Publishing, 2014.

%[2] G. Borowik, K. Kowalski: Efektywna procedura uzupełnienia funkcji boolowskich i jej zastosowaniew eksploracji danych. Przegląd Telekomunikacyjny i Wiadomości Telekomunikacyjne. 03/2015.
\bibitem{efektywna-procedura} G. Borowik, K. Kowalski \emph{,,Efektywna procedura uzupełnienia funkcji boolowskich i jej zastosowanie w eksploracji danych''}, Przegląd Telekomunikacyjny i Wiadomości Telekomunikacyjne, 03/2015.

\bibitem{inzynierka} K. Kowalski \emph{,,Implementacja Algorytmu Uzupełnienia Funkcji Boolowskich z Wykorzystaniem Programowania Współbieżnego''}, Praca Dyplomowa Inżynierska, 2014.

%[3] Chen, D., Wang, C., Hu, Q., 2007. A new approach to attribute reduction of consistent and inconsistent covering decision systems with covering rough sets. Information Sciences 177, 3500–3518.
\bibitem{new-reduction} D. Chen, C. Wang, Q. Hu \emph{,,A new approach to attribute reduction of consistent and inconsistent covering decision systems with covering rough sets''}, Information Sciences 177, 3500–3518, 2007.

%[4] Łuba T., Borowik G.: Synteza logiczna, Oficyna Wydawnicza PW, Warszawa 2015.
\bibitem{synteza-logiczna} T. Łuba, G. Borowik \emph{,,Synteza logiczna''}, Oficyna Wydawnicza PW, Warszawa 2015.

%[5] Łuba T., Poźniak K., Zbierzchowski B.: Redukcja i kompresja zmiennych w syntezie funkcji generowania indeksów. Przegląd telekomunikacyjny, nr. 10, 2016.
\bibitem{redukcja-kompresja} T. Łuba, K. Poźniak, B. Zbierzchowski \emph{,,Redukcja i kompresja zmiennych w syntezie funkcji generowania indeksów''}, Przegląd telekomunikacyjny, nr. 10, 2016.

%[6] Łuba, T., Rybnik J.:  Algorithmic Approach to Discernibility Function with Respect to Attributes and Objects Reduction, Foundations of Computing and Decision Sciences, Vol. 18, No. 3-4, 241–258, 1993.
\bibitem{algorithmic-approach} T. Łuba, J. Rybnik \emph{,,Algorithmic Approach to Discernibility Function with Respect to Attributes and Objects Reduction''}, Foundations of Computing and Decision Sciences, Vol. 18, No. 3-4, 241–258, 1993.

%[7] Sasao T.: Index Generation Functions, Logic Synthesis for Pattern Matching, EPFL Workshop on Logic Synthesis & Verification, Dec. 2015.
\bibitem{sasao-workshop} T. Sasao \emph{,,Index Generation Functions''}, Logic Synthesis for Pattern Matching, EPFL Workshop on Logic Synthesis \& Verification, Dec. 2015.

%[8] Sasao, T.: Index generation functions: Recent developments,  International Symposium on Multiple-Valued Logic (ISMVL-2011), Tuusula, Finland, May 2011.
\bibitem{sasao-recent} T. Sasao \emph{,,Index generation functions: Recent developments''}, International Symposium on Multiple-Valued Logic (ISMVL-2011), Tuusula, Finland, May 2011.

%[9] Sasao, T.:  A Reduction Method For the Number of Variables to Represent Index Generation Functions: s-Min Method, IEEE 45th International Symposium on Multiple-Valued Logic, 164–169, 2015.
\bibitem{sasao-s-min} T. Sasao \emph{,,A Reduction Method For the Number of Variables to Represent Index Generation Functions: s-Min Method''}, IEEE 45th International Symposium on Multiple-Valued Logic, 164–169, 2015.

%[10] Sasao, T.: Memory-Based Logic Synthesis, Springer New York Dordrecht Heidelberg London, 2011.
\bibitem{sasao-synthesis} T. Sasao \emph{,,Memory-Based Logic Synthesis''}, Springer New York Dordrecht Heidelberg London, 2011.

%[11] RSES – Rough Set Exploration System, http://logic.mimuw.edu.pl/~rses/
\bibitem{rses} \emph{,,RSES – Rough Set Exploration System''} [Online] http://logic.mimuw.edu.pl/~rses/

%[12] Steinbach B., Posthoff C., Improvements of the Construction of Exact Minimal Covers of Boolean Functions, Computer Aided Systems Theory – EUROCAST 2011, Lecture Notes in Computer Science Volume 6928, pp. 272-279, 2012.
\bibitem{steinbach-posthoff} B. Steinbach, C. Posthoff \emph{,,Improvements of the Construction of Exact Minimal Covers of Boolean Functions''}, Computer Aided Systems Theory – EUROCAST 2011, Lecture Notes in Computer Science Volume 6928, pp. 272-279, 2012.

%[13] Skowron, A., Rauszer, C., 1992. The discernibility matrices and functions in information systems, in: Słowiński, R. (Ed.), Intelligent Decision Support. Springer Netherlands. volume 11 of Theory and Decision Library, pp. 331– 362.
\bibitem{skowron-rauszer} A. Skowron, C. Rauszer \emph{,,The discernibility matrices and functions in information systems''}, Słowiński, R. (Ed.), Intelligent Decision Support. Springer Netherlands. volume 11 of Theory and Decision Library, pp. 331– 362, 1992.

%[14] Ślezak, D., 1998. Searching for dynamic reducts in inconsistent decision tables,  in: Proceedings of IPMU 98, pp. 1362–1369.
\bibitem{slezak} D. Ślezak \emph{,,Searching for dynamic reducts in inconsistent decision tables''}, Proceedings of IPMU 98, pp. 1362–1369. 1998.

%[15] Wang, C., He, Q., Chen, D., Hu, Q., 2014. A novel method for attribute reduction of covering decision systems. Information Sciences 254, 181–196.
\bibitem{novel-method} C. Wang, Q. He, D. Chen, Q. Hu \emph{,,A novel method for attribute reduction of covering decision systems''}, Information Sciences 254, 181–196, 2014.

\bibitem{unate-artykul} T. Łuba, G. Borowik, K. Kowalski, P. Pecio, C. Jankowski, M. Mańkowski \emph{,,Rola i znaczenie syntezy logicznej w eksploracji danych dla potrzeb telekomunikacji i medycyny''}, Przegląd Telekomunikacyjny, str. 110-116, 05/2014.

\bibitem{without-matrix} M. Korzeń, S. Jaroszewicz \emph{,,Finding Reducts Without Building the Discernibility Matrix''}, Intelligent Systems Design and Applications, 2005.

\end{thebibliography}
\clearpage




%===============================================================================
							% bibliografia

%===============================================================================
% Koniec

%*******************************************************************************
% Bibliografia - spis literatury wykorzystanej przy tworzeniu pracy
%*******************************************************************************

\begin{thebibliography}{99}
\addcontentsline{toc}{chapter}{Bibliografia}
\bibliographystyle{tran}

%[1] Borowik G., Łuba T.: Fast Algorithm of Attribute Reduction Based on the complementation of Boolean Function, ch. 2, pp. 25-41, Springer International Publishing, 2014.
\bibitem{fast-algorithm} G. Borowik, T. Łuba \emph{,,Fast Algorithm of Attribute Reduction Based on the complementation of Boolean Function''}, ch. 2, pp. 25-41, Springer International Publishing, 2014.

%[2] G. Borowik, K. Kowalski: Efektywna procedura uzupełnienia funkcji boolowskich i~jej zastosowaniew eksploracji danych. Przegląd Telekomunikacyjny i~Wiadomości Telekomunikacyjne. 03/2015.
\bibitem{efektywna-procedura} G. Borowik, K. Kowalski \emph{,,Efektywna procedura uzupełnienia funkcji boolowskich i~jej zastosowanie w~eksploracji danych''}, Przegląd Telekomunikacyjny i~Wiadomości Telekomunikacyjne, 03/2015.

\bibitem{inzynierka} K. Kowalski \emph{,,Implementacja Algorytmu Uzupełnienia Funkcji Boolowskich z~Wykorzystaniem Programowania Współbieżnego''}, Praca Dyplomowa Inżynierska, 2014.

%[3] Chen, D., Wang, C., Hu, Q., 2007. A~new approach to attribute reduction of consistent and inconsistent covering decision systems with covering rough sets. Information Sciences 177, 3500–3518.
\bibitem{new-reduction} D. Chen, C. Wang, Q. Hu \emph{,,A new approach to attribute reduction of consistent and inconsistent covering decision systems with covering rough sets''}, Information Sciences 177, 3500–3518, 2007.

%[4] Łuba T., Borowik G.: Synteza logiczna, Oficyna Wydawnicza PW, Warszawa 2015.
\bibitem{synteza-logiczna} T. Łuba, G. Borowik \emph{,,Synteza logiczna''}, Oficyna Wydawnicza PW, Warszawa 2015.

%[5] Łuba T., Poźniak K., Zbierzchowski B.: Redukcja i~kompresja zmiennych w~syntezie funkcji generowania indeksów. Przegląd telekomunikacyjny, nr. 10, 2016.
\bibitem{redukcja-kompresja} T. Łuba, K. Poźniak, B. Zbierzchowski \emph{,,Redukcja i~kompresja zmiennych w~syntezie funkcji generowania indeksów''}, Przegląd telekomunikacyjny, nr. 10, 2016.

%[6] Łuba, T., Rybnik J.:  Algorithmic Approach to Discernibility Function with Respect to Attributes and Objects Reduction, Foundations of Computing and Decision Sciences, Vol. 18, No. 3-4, 241–258, 1993.
\bibitem{algorithmic-approach} T. Łuba, J. Rybnik \emph{,,Algorithmic Approach to Discernibility Function with Respect to Attributes and Objects Reduction''}, Foundations of Computing and Decision Sciences, Vol. 18, No. 3-4, 241–258, 1993.

%[7] Sasao T.: Index Generation Functions, Logic Synthesis for Pattern Matching, EPFL Workshop on Logic Synthesis & Verification, Dec. 2015.
\bibitem{sasao-workshop} T. Sasao \emph{,,Index Generation Functions''}, Logic Synthesis for Pattern Matching, EPFL Workshop on Logic Synthesis \& Verification, Dec. 2015.

%[8] Sasao, T.: Index generation functions: Recent developments,  International Symposium on Multiple-Valued Logic (ISMVL-2011), Tuusula, Finland, May 2011.
\bibitem{sasao-recent} T. Sasao \emph{,,Index generation functions: Recent developments''}, International Symposium on Multiple-Valued Logic (ISMVL-2011), Tuusula, Finland, May 2011.

%[9] Sasao, T.:  A~Reduction Method For the Number of Variables to Represent Index Generation Functions: s-Min Method, IEEE 45th International Symposium on Multiple-Valued Logic, 164–169, 2015.
\bibitem{sasao-s-min} T. Sasao \emph{,,A Reduction Method For the Number of Variables to Represent Index Generation Functions: s-Min Method''}, IEEE 45th International Symposium on Multiple-Valued Logic, 164–169, 2015.

%[10] Sasao, T.: Memory-Based Logic Synthesis, Springer New York Dordrecht Heidelberg London, 2011.
\bibitem{sasao-synthesis} T. Sasao \emph{,,Memory-Based Logic Synthesis''}, Springer New York Dordrecht Heidelberg London, 2011.

%[11] RSES – Rough Set Exploration System, http://logic.mimuw.edu.pl/~rses/
\bibitem{rses} \emph{,,RSES – Rough Set Exploration System''} [Online] http://logic.mimuw.edu.pl/~rses/

%[12] Steinbach B., Posthoff C., Improvements of the Construction of Exact Minimal Covers of Boolean Functions, Computer Aided Systems Theory – EUROCAST 2011, Lecture Notes in Computer Science Volume 6928, pp. 272-279, 2012.
\bibitem{steinbach-posthoff} B. Steinbach, C. Posthoff \emph{,,Improvements of the Construction of Exact Minimal Covers of Boolean Functions''}, Computer Aided Systems Theory – EUROCAST 2011, Lecture Notes in Computer Science Volume 6928, pp. 272-279, 2012.

%[13] Skowron, A., Rauszer, C., 1992. The discernibility matrices and functions in information systems, in: Słowiński, R. (Ed.), Intelligent Decision Support. Springer Netherlands. volume 11 of Theory and Decision Library, pp. 331– 362.
\bibitem{skowron-rauszer} A. Skowron, C. Rauszer \emph{,,The discernibility matrices and functions in information systems''}, Słowiński, R. (Ed.), Intelligent Decision Support. Springer Netherlands. volume 11 of Theory and Decision Library, pp. 331– 362, 1992.

%[14] Ślezak, D., 1998. Searching for dynamic reducts in inconsistent decision tables,  in: Proceedings of IPMU 98, pp. 1362–1369.
\bibitem{slezak} D. Ślezak \emph{,,Searching for dynamic reducts in inconsistent decision tables''}, Proceedings of IPMU 98, pp. 1362–1369. 1998.

%[15] Wang, C., He, Q., Chen, D., Hu, Q., 2014. A~novel method for attribute reduction of covering decision systems. Information Sciences 254, 181–196.
\bibitem{novel-method} C. Wang, Q. He, D. Chen, Q. Hu \emph{,,A novel method for attribute reduction of covering decision systems''}, Information Sciences 254, 181–196, 2014.

\bibitem{unate-artykul} T. Łuba, G. Borowik, K. Kowalski, P. Pecio, C. Jankowski, M. Mańkowski \emph{,,Rola i~znaczenie syntezy logicznej w~eksploracji danych dla potrzeb telekomunikacji i~medycyny''}, Przegląd Telekomunikacyjny, str. 110-116, 05/2014.

\bibitem{without-matrix} M. Korzeń, S. Jaroszewicz \emph{,,Finding Reducts Without Building the Discernibility Matrix''}, Intelligent Systems Design and Applications, 2005.

\bibitem{memory-capacity} G. Borowik, T. Łuba, P. Tomaszewicz \emph{,,On the Memory Capacity to Implement Logic Functions''}, Eds. F. Pichler, R. Moreno-Díaz, A. Quesada-Arencibia, Computer Aided Systems Theory – EUROCAST 2011, vol.6928: Springer-Verlag Berlin Heidelberg, Lecture Notes in Computer Science, 343-350, 2012.

\bibitem{nine-filters} D.J.Goodman, M.J. Carey \emph{,,Nine Digital Filters for Decimation and Interpolation''}, IEEE Trans. On Acoustics, Speech and Signal Processing 25(2), pp.121-126, 1977.

\bibitem{redukcja-kompresja} T. Łuba, K. Poźniak, B. Zbierzchowski \emph{,,Redukcja i kompresja zmiennych w syntezie funkcji generowania indeksów''}, Przegląd Telekomunikacyjny i Wiadomości Telekomunikacyjne, pp. 1230 - 1236, Nr. 10/2016.

\bibitem{wirusy} H. Nakahara, T. Sasao, M. Matsuura, Y. Kawamura, \emph{,,A parallel sieve method for a virus scanning engine''}, 12th EUROMICRO Conference on Digital System Design, Architectures, Methods and Tools, Patras, Greece (DSD-2009), Aug. 2009, pp.809-816.

\end{thebibliography}
\clearpage




%===============================================================================


\chapter*{Wykaz symboli i skrótów}
\addcontentsline{toc}{chapter}{Wykaz symboli i skrótów}

\noindent
\textbf{F} - funkcja, \newline
\textbf{G} - wejściowa składowa dekompozycji, \newline
\textbf{H} - wyjsciowa składowa dekompozycji, \newline
\textbf{k} - liczba wierszy funkcji, \newline
\textbf{MR} - macierz rozróżnialności, \newline
\textbf{m} - liczba wierszy funkcji, \newline
\textbf{n} - liczba argumentów funkcji, \newline
\textbf{R} - redukt, \newline
\textbf{\texorpdfstring{w\textsubscript{i}}}{ - i-ty wiersz funckji,}\newline
\textbf{X} - zbiór argumentów, \newline
\textbf{\texorpdfstring{x\textsubscript{i}}{x i}} - i-ty argument funckji, \newline
\textbf{\texorpdfstring{x\textsubscript{i\textsubscript{w\textsubscript{j}}}}{x i w j}} - wartość i-tego argumentu dla j-tego wiersza w funkcji, \newline
\textbf{Y} - zbiór wartości funkcji, \newline
\textbf{\texorpdfstring{y\textsubscript{i}}{y i}} - i-ta wrtość funkcji, \newline

\cleardoublepage

\listoffigures
\addcontentsline{toc}{chapter}{Spis rysunków}
\cleardoublepage

\listoftables
\addcontentsline{toc}{chapter}{Spis tablic}
\cleardoublepage

%Spis załączników

%Załączniki

\end{document}

%===============================================================================
