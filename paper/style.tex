%\documentclass[12pt, a4paper, oneside, titlepage, final]{book}
\documentclass[12pt, a4paper, twoside, titlepage, final]{book}
\renewcommand{\familydefault}{\sfdefault}
%\documentclass[12pt,a4paper,onecolumn,oneside,11pt,wide,floatssmall]{book}


%===============================================================================
% Pakiety
\usepackage{polski}
\usepackage[utf8]{inputenc}

\usepackage[OT4]{fontenc}					% font PL
%\usepackage{url}								% polecenie \url
\usepackage{amsfonts}						% fonty matematyczne
\usepackage{graphicx}						% wstawianie grafiki
%\usepackage{listings}						% wstawianie kodu
\usepackage{color}							% kolory
\usepackage{fancyhdr}						% paginy g�rne i dolne
\usepackage[plainpages=false]{hyperref}% dynamiczne linki
%\usepackage{calc}								% operacje arytmetyczne w TeX'u
%\usepackage{tabularx}						% rozci�gliwe tabele
%\usepackage{array}							% standardowe tabele
\usepackage{graphicx}
\usepackage{geometry}
\usepackage{hyperref}
\hypersetup{
    colorlinks,
    citecolor=black,
    filecolor=black,
    linkcolor=black,
    urlcolor=blue
}
\linespread{1.3}								% 1.3 do interlinii 1.5


% w�asne pakiety
\usepackage{pwtitle}

%===============================================================================
% Ustawienia dokumentu

\frenchspacing

% ustawienia wymiar�w


\oddsidemargin 20mm							% margines nieparzystych stron
\evensidemargin 20mm							% margines parzystych stron
%\headheight 15pt								% wysoko�� paginy g�rnej
%\topmargin 20mm									% margines g�rny

% styl paginacji
\pagestyle{fancy}
%\renewcommand{\chaptermark}[1]{\markboth{#1}{}}
%\renewcommand{\sectionmark}[1]{\markright{\thesection\ #1}{}}

% nag��wek 
\fancyhf{}
\fancyfoot[LE,RO]{\thepage}
\fancyhead[C]{\small\nouppercase{\rightmark}}

%\fancyhead[R]{\small\nouppercase{\leftmark}}
\renewcommand{\headrulewidth}{0.1pt}
\renewcommand{\footrulewidth}{0pt}

% nag��wek w stylu plain 
\fancypagestyle{plain}
{
\fancyhf{}
\renewcommand{\headrulewidth}{0pt}
\renewcommand{\footrulewidth}{0pt}
}

% ta sekwencja tworzy czyste kartki na stronach po \cleardoublepage
\makeatletter
\def\cleardoublepage{\clearpage\if@twoside \ifodd\c@page\else
	\hbox{}
	\vspace*{\fill}
	\thispagestyle{empty}
	\newpage
	\if@twocolumn\hbox{}\newpage\fi\fi\fi}
\makeatother

%===============================================================================
% Zmienne �rodowiskowe i polecenia

% definicja
\newtheorem{definition}{Definicja}[chapter]

% twierdzenie
\newtheorem{theorem}{Twierdzenie}[chapter]

% obcoj�zyczne nazwy
\newcommand{\foreign}[1]{\emph{#1}}

% pozioma linia
\newcommand{\horline}{\noindent\rule{\textwidth}{0.4mm}}

% wstawianie obrazk�w {plik}{caption}{opis}


%===============================================================================
% ustawienia pakietu hyperref

\hypersetup
{
%colorlinks=true,			% false: boxed links; true: colored links
%linkcolor=black,			% color of internal links
%citecolor=black,			% color of links to bibliography
%filecolor=black,			% color of file links
%urlcolor=black			% color of external links
}
